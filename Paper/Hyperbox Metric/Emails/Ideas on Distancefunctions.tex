% Options for packages loaded elsewhere
\PassOptionsToPackage{unicode}{hyperref}
\PassOptionsToPackage{hyphens}{url}
%
\documentclass[
]{article}
\usepackage{amsmath,amssymb}
\usepackage{iftex}
\ifPDFTeX
  \usepackage[T1]{fontenc}
  \usepackage[utf8]{inputenc}
  \usepackage{textcomp} % provide euro and other symbols
\else % if luatex or xetex
  \usepackage{unicode-math} % this also loads fontspec
  \defaultfontfeatures{Scale=MatchLowercase}
  \defaultfontfeatures[\rmfamily]{Ligatures=TeX,Scale=1}
\fi
\usepackage{lmodern}
\ifPDFTeX\else
  % xetex/luatex font selection
\fi
% Use upquote if available, for straight quotes in verbatim environments
\IfFileExists{upquote.sty}{\usepackage{upquote}}{}
\IfFileExists{microtype.sty}{% use microtype if available
  \usepackage[]{microtype}
  \UseMicrotypeSet[protrusion]{basicmath} % disable protrusion for tt fonts
}{}
\makeatletter
\@ifundefined{KOMAClassName}{% if non-KOMA class
  \IfFileExists{parskip.sty}{%
    \usepackage{parskip}
  }{% else
    \setlength{\parindent}{0pt}
    \setlength{\parskip}{6pt plus 2pt minus 1pt}}
}{% if KOMA class
  \KOMAoptions{parskip=half}}
\makeatother
\usepackage{xcolor}
\setlength{\emergencystretch}{3em} % prevent overfull lines
\providecommand{\tightlist}{%
  \setlength{\itemsep}{0pt}\setlength{\parskip}{0pt}}
\setcounter{secnumdepth}{-\maxdimen} % remove section numbering
\ifLuaTeX
  \usepackage{selnolig}  % disable illegal ligatures
\fi
\IfFileExists{bookmark.sty}{\usepackage{bookmark}}{\usepackage{hyperref}}
\IfFileExists{xurl.sty}{\usepackage{xurl}}{} % add URL line breaks if available
\urlstyle{same}
\hypersetup{
  hidelinks,
  pdfcreator={LaTeX via pandoc}}

\author{}
\date{}

\begin{document}

Given are the two distance functions, which shall be discussed in this
short note. Let \((X,d_{X})\) be a metric space with
\(A,B \in \mathcal{P}(X)\) and \(x \in X\), then we will discuss the
following distance functions. 1.
\(d(x,A) = \underset{a \in A}{\text{inf}}\; d_{X}(x,a)\) (Point Set
Distance) 2.
\(h(A,B) = \underset{}{\text{max}}\;\{ \underset{a \in A}{\text{sup}}\;\underset{b \in B}{\text{inf}\vphantom{\text{p}}}\;d_{X}(a,b), \underset{b \in B}{\text{sup}}\;\underset{a \in A}{\text{inf}\vphantom{\text{p}}}\;d_{X}(a,b)\}\)

\hypertarget{definitions}{%
\subsection{Definitions}\label{definitions}}

\hypertarget{point-set-distance}{%
\subsubsection{Point Set Distance}\label{point-set-distance}}

\textbf{Definition Point Set Distance} Let - \((X,d_{X})\) be a metric
space - \(\mathcal{P}(X)\) the power set of \(X\) - \(x \in X\)\\
- \(A \in \mathcal{P}(X)\setminus \emptyset\)

then the distance between \(x\) and \(A\) can be given as
\[\begin{align} 
 d: X \times \mathcal{P}(X)\setminus \emptyset \to \mathbb{R} 
\end{align}\]\\
with \[ d(x,A) = \underset{a \in A}{\text{inf}} \; d_{X}(x,a)\;\]

\textbf{Remark Metric Axioms} \[\begin{alignat}{2}
(i)& \quad&&d(x,y) = d(y,x) \tag{Symmetry} \\
(ii)& \;\quad&&d(x,y) \geq 0 \text{ and } d(x,x)=0 \tag{Non-Negativeness} \\
(iii)& \quad &&d(x,y)=0 \Rightarrow x=y \tag{Positiveness} \\ 
(iv)& \quad &&d(x,y) \leq d(x,z) + d(z,y) \tag{Triangle Inequality}
\end{alignat}\] - Symmetry ❌
\[d(x,A)=\underset{a \in A}{\text{inf}}\; d_{X}(a,x)\neq d(A,x)\] Note,
the function is defined such that the first argument is a point and the
second argument is a set

\begin{itemize}
\item
  Non-Negativeness ✅
  \[d(A,x)=\underset{a \in A}{\text{inf}}\; \underbrace{ d_{X}(a,x) }_{ \geq 0 }\]
  Since \(d_{X}\) is a metric and is guaranteed to be non-negative, the
  infimum must be also non-negative
\item
  Positiveness ❌ Clearly \(x \neq A\) since \(x \in X\) and
  \(A \in \mathcal{P}(X)\). Further, there are potentially infinitely
  many \(A \in \mathcal{P}(X)\) that hold \(d(x,A) = 0\), i.e.~let
  \(A_{1}=[0,3]\), \(A_{2}=[1,2]\) and \(x=1\), then \[\begin{align}
  d(x,A_{1}) &= \underset{a \in A_{1}}{\text{inf}}\; d(a,x) = 0 \\ 
  d(x,A_{2}) &= \underset{a \in A_{2}}{\text{inf}}\; d(a,x) = 0
  \end{align}\]
\item
  Triangle Inequality ❌ Again, since \(x \in X\) and
  \(A \in \mathcal{\mathcal{P}}(X)\), we have the problem of properly
  setting up the triangle inequality, i.e.~let \(b \in X\) and
  \(B \in \mathcal{P}(X)\), then we have \[\begin{align}
  d(x,A) &\leq d(x,B) + \underbrace{ d(B,A) }_{ \text{not defined} } \\
  d(A,x) &\leq d(A,b) + \underbrace{ d(b,x) }_{ \text{not defined} } 
  \end{align}\]
\end{itemize}

We can extend the concept of the \textbf{point set distance} to a
\textbf{ordinary set to set distance} on the same concept

\textbf{Definition Ordinary Set-to-Set Distance} Let - \((X,d_{X})\) be
a metric space - \(\mathcal{P}(X)\) the power set of \(X\) -
\(A,B \in \mathcal{P}(X)\setminus \emptyset\)

then the distance between A and B can be given as
\[d: \mathcal{P}(X)\setminus \emptyset \times \mathcal{P}(X)\setminus \emptyset \to \mathbb{R}\]
with \[d(A,B) = \underset{a \in A, b \in B}{\text{inf}}\; d_{X}(a,b)\]

\textbf{Remark Metric Axioms} \[\begin{alignat}{2}
(i)& \quad&&d(x,y) = d(y,x) \tag{Symmetry} \\
(ii)& \;\quad&&d(x,y) \geq 0 \text{ and } d(x,x)=0 \tag{Non-Negativeness} \\
(iii)& \quad &&d(x,y)=0 \Rightarrow x=y \tag{Positiveness} \\ 
(iv)& \quad &&d(x,y) \leq d(x,z) + d(z,y) \tag{Triangle Inequality}
\end{alignat}\] - Symmetry ✅
\[d(A,B)=\underset{a \in A, b \in B}{\text{inf}}\; d_{X}(a,b)= d(B,A)\]

\begin{itemize}
\item
  Non-Negativeness ✅
  \[d(A,B)=\underset{a \in A, b \in B}{\text{inf}}\; \underbrace{ d_{X}(a,b) }_{ \geq 0 }\]
  Since \(d_{X}\) is a metric and is guaranteed to be non-negative, the
  infimum must be also non-negative
\item
  Positiveness ❌ For a similar reason as in the
  point to set distance, we can
  easily construct cases, such that \(d \equiv 0\) for distinct sets
  \(A,B\). More precisely, any sets \(A,B\) with
  \(A \cap B \neq \emptyset\) have \[d(A,B) = 0\] The following example
  illustrates the problem. Let \(A = [0,2]\) and \(B = [1,3]\), then
  note that \(A \cap B = [1,2]\). For the infimum in
  \(\underset{a \in A, b \in B}{\text{inf}}\;d(a,b)\) we can choose any
  \(a,b \in A \cap B\) and get \[d(A,B) = 0\] with \(A \neq B\)
\item
  Triangle Inequality ❌ The following example will illustrate the issue
  with this property. Let again \(A = [0,2]\), \(B=[1,5]\), \(C=[4,6]\)
  and \(d_X\) the \(p=1\) Minkowski distance, we have
  \[\begin{alignat}{2}
  d(A,C) &= \underset{a \in A, c \in C}{\text{inf}}\;d_{X}(a,c) &&= d_{X}(2,4) = 2 \\
  d(A,B) &= \underset{a \in A, b \in B}{\text{inf}}\;d_{X}(a,b) &&= d_{X}(1,1) = 0 \\
  d(B,C) &= \underset{b \in B, c \in C}{\text{inf}}\;d_{X}(b,c) &&= d_{X}(4,4) = 0 \\
  \end{alignat}\] Then we have for the triangle inequality
  \[\begin{alignat}{2}
  d(A,C) &\leq d(A,B) &&+ d(B,C) \\
  2 &\not\leq 0 &&+ 0
  \end{alignat}\]
\end{itemize}

\hypertarget{general-hausdorff-function}{%
\subsubsection{General Hausdorff
Function}\label{general-hausdorff-function}}

\textbf{Definition The Hausdorff Family}

Let - \((X,d_{X})\) be a metric space - \(\mathcal{P}(X)\) the power set
of \(X\)

The Hausdorff function
\[h: \mathcal{P}(X)\setminus \emptyset \times \mathcal{P}(X)\setminus \emptyset \to \overline{\mathbb{R}}\]
is given \(\forall A, B \in \mathcal{P}(X) \setminus \emptyset\)
\[\begin{alignat}{2}
 h(A,B) &= \underset{}{\text{max}}\;\{ \underset{a \in A}{\text{sup}}\;d_{X}(a,B)&&, \underset{b \in B}{\text{sup}} \; d_{X}(b,A)\}\\
        &= \underset{}{\text{max}}\;\{ \underset{a \in A}{\text{sup}}\;\underset{b \in B}{\text{inf}\vphantom{\text{p}}}\;d_{X}(a,b)&&, \underset{b \in B}{\text{sup}}\;\underset{a \in A}{\text{inf}\vphantom{\text{p}}}\;d_{X}(a,b)\}
\end{alignat}\]

\textbf{Remark Axiom Metrics} \[\begin{alignat}{2}
(i)& \quad&&d(x,y) = d(y,x) \tag{Symmetry} \\
(ii)& \;\quad&&d(x,y) \geq 0 \text{ and } d(x,x)=0 \tag{Non-Negativeness} \\
(iii)& \quad &&d(x,y)=0 \Rightarrow x=y \tag{Positiveness} \\ 
(iv)& \quad &&d(x,y) \leq d(x,z) + d(z,y) \tag{Triangle Inequality}
\end{alignat}\] - Symmetry ✅ The Hausdorff function satisfies the
symmetry property, since
\[d(A,B) = \underset{}{\text{max}}\{ \underset{a \in A}{\text{sup}} \; d_{X}(a,B), \underset{b \in B}{\text{sup}} \; d_{X}(b, A)\}\; = d(B,A)\]
The suprema inside the maximum operator are interchangeable and the
maximum value will be returned regardless of the order the arguments are
passed into the distance function.

\begin{itemize}
\item
  Non-Negativeness ✅ Since the Hausdorff function relies on the
  underlying metric space \(X\) and its distance function, the values
  are guaranteed to be at least 0
\item
  Remaining Ideas ❓ It is not a distance function, since it maps to the
  extended real numbers \(\overline{\mathbb{R}}\) with
  \[h(\{ 0 \}, \mathbb{R}) = + \infty\] It is not a pseudo metric,
  because it does not satisfy property \[\begin{alignat}{2}
   (iii)& \quad &&d(x,y)=0 \Longrightarrow x=y \tag{positiveness}
  \end{alignat}\] because let \(A=(0,1)\) and \(B=[0,1]\), we have
  \[h(A,B)=0\]
\end{itemize}

Therefore, we have to require compact subsets as domain of the distance
function, i.e.~
\[h: \mathcal{B}(X)\setminus \emptyset \times \mathcal{B}(X)\setminus \emptyset \to \mathbb{R}\]

\hypertarget{conclusion}{%
\subsection{Conclusion}\label{conclusion}}

The point-set distance and the set-to-set distance don't seem to be
stable options for a proper learning process, since they fail to hold
many of the metric properties.

The Hausdorff metric on the other hand seems counter-intuitive for the
objective of the learning process. Given the Hyperbox-LVQ method, we
could summarize the objective for an optimal prototype placement and
size as follows: 1. a box prototype should just be as large as needed to
cover a cloud of data points 2. overlaps between box prototypes should
be as small as possible, which should result from the 1. 3. the position
of the center of the box should allow point 1. and 2. to hold

Since the Hausdorff distance describes the maximum distance between a
closest point of a set, and the farthest point of the other set with
respect to the closest point, it would incentivise the collapse of the
boxes over time, such that they collapse on data points, because then
the distance would be minimal, i.e.~0.

Since the set, i.e.~the hyperbox, is described over its center and its
dimensions, it seems natural to focus on designing a distance function
involving these two parameters.

\hypertarget{ideas}{%
\subsubsection{Ideas}\label{ideas}}

Since there are two distinct objectives in optimizing prototypes, an
alternating learning scheme seems an appropriate method. 1. Placement of
the centers could follow the box-distance \(d_B\) (needs to be checked,
only idea for now)
\[d_{B}(\mathbf{w},\mathbf{x})= d(\mathbf{w}, \mathbf{x}) \cdot \sigma(\text{Overlap-Factor}(\mathbf{w},\mathbf{x}))\]
where the overlap factor is the magnitude of overlap between the two
boxes of \(\mathbf{w}, \mathbf{x}\).
\[\text{Overlap-Factor}(\mathbf{w}, \mathbf{x})=\begin{cases}
  \>>0 & \text{ if boxes overlap}\\ 
  =0 & \text{ if boxes touch} \\
<0 & \text{ else }
\end{cases}\] 2. Optimizing the size of the boxes. After placing the
boxes, the algorithm should check if there are data points
\(\mathbf{x}\) which are not yet covered by a box of same class. The
suggestion follows - \textbf{If} there are points which are not covered
by any box of the same class yet =\textgreater{} the size of the box
should be increased. - \textbf{If} there are all prototypes are covered
by a box of the same class - \textbf{if} the farthest data point inside
the box is exactly on the border of the prototype =\textgreater{} the
box should neither decrease or increase. - \textbf{If} the farthest data
point inside the box is inside the box of the prototype =\textgreater{}
the box should decrease

\end{document}

% Options for packages loaded elsewhere
\PassOptionsToPackage{unicode}{hyperref}
\PassOptionsToPackage{hyphens}{url}
%
\documentclass[
]{article}
\usepackage{amsmath,amssymb}
\usepackage{iftex}
\ifPDFTeX
  \usepackage[T1]{fontenc}
  \usepackage[utf8]{inputenc}
  \usepackage{textcomp} % provide euro and other symbols
\else % if luatex or xetex
  \usepackage{unicode-math} % this also loads fontspec
  \defaultfontfeatures{Scale=MatchLowercase}
  \defaultfontfeatures[\rmfamily]{Ligatures=TeX,Scale=1}
\fi
\usepackage{lmodern}
\ifPDFTeX\else
  % xetex/luatex font selection
\fi
% Use upquote if available, for straight quotes in verbatim environments
\IfFileExists{upquote.sty}{\usepackage{upquote}}{}
\IfFileExists{microtype.sty}{% use microtype if available
  \usepackage[]{microtype}
  \UseMicrotypeSet[protrusion]{basicmath} % disable protrusion for tt fonts
}{}
\makeatletter
\@ifundefined{KOMAClassName}{% if non-KOMA class
  \IfFileExists{parskip.sty}{%
    \usepackage{parskip}
  }{% else
    \setlength{\parindent}{0pt}
    \setlength{\parskip}{6pt plus 2pt minus 1pt}}
}{% if KOMA class
  \KOMAoptions{parskip=half}}
\makeatother
\usepackage{xcolor}
\setlength{\emergencystretch}{3em} % prevent overfull lines
\providecommand{\tightlist}{%
  \setlength{\itemsep}{0pt}\setlength{\parskip}{0pt}}
\setcounter{secnumdepth}{-\maxdimen} % remove section numbering
\ifLuaTeX
  \usepackage{selnolig}  % disable illegal ligatures
\fi
\IfFileExists{bookmark.sty}{\usepackage{bookmark}}{\usepackage{hyperref}}
\IfFileExists{xurl.sty}{\usepackage{xurl}}{} % add URL line breaks if available
\urlstyle{same}
\hypersetup{
  hidelinks,
  pdfcreator={LaTeX via pandoc}}

\author{}
\date{}

\begin{document}

\subsubsection{Kombinatorik}\label{kombinatorik}

\paragraph{Aufgabe (1a)}\label{aufgabe-1a}

\begin{quote}
Bemerke, wir haben 6 \textbf{verschiedene} Personen, die wir in allen
möglichen Anordnungen ablichten lassen wollen. Da keine Person mehrfach
auftauchen kann, können wir auf eine \textbf{Permutation ohne
Wiederholungen} schließen und wenden die Formel unter \textbf{Defintion
1.1} an. Wir haben \[\begin{align}
P_{n}=n! \quad \overset{n=6}{\longrightarrow} \quad P_{6} = 6! = 720
\end{align}
\] Somit sind \(720\) Anordnungen möglich
\end{quote}

\paragraph{Aufgabe (1b)}\label{aufgabe-1b}

\begin{quote}
Wie in Aufgabe (1a) suchen wir alle möglichen Anordnungen, die durch **
3 Paare eineiiger Zwillinge** realisiert werden können. Das Attribut der
\textbf{eineiigen Zwillinge} deutet an, dass diese auf dem Foto als
identisch betrachtet werden können. Um den Sacherverhalt darzustellen
wenden wir \textbf{Definition 1.3 (Multimenge)} an \[\begin{align}
M = \{ \underbrace{\overbrace{1,1}}^{\text{Paar 1}}_{2-\text{mal}},\underbrace{\overbrace{2,2}}^{\text{Paar 2}}_{2-\text{mal}},\underbrace{\overbrace{3,3}}^{\text{Paar 3}}_{2-\text{mal}} \}
\end{align}\] Unter dieser Darstellung können wir einfach
\textbf{Definition 1.4} anwenden. Mit \[\begin{align}
n&=6 \\
k_{1-3} &= 2
\end{align}\] erhalten wir \[\begin{align}
\binom{n}{k_{1},k_{2}, \ldots, k_{r}}&= \frac{n!}{k_{1}!k_{2}!\dots k_{r}!} \\
&= \frac{6!}{2!2!2!} \\
&= \frac{720}{2 \cdot 2 \cdot 2} \\
&= 90
\end{align}\]
\end{quote}

\paragraph{Aufgabe (1c)}\label{aufgabe-1c}

\begin{quote}
Wir bemerken, dass genau 3 aus 10 gesucht sind. Daher können wir
feststellen, dass wir nicht Anordnungen einer Gesamtmenge, sondern
Varianten (Variationen) oder auch Kombinationen aus einer
\textbf{Auswahl} suchen.

Um zwischen \textbf{Kombination und Variation} zu unterscheiden, müssen
wir uns überlegen ob eine Ordnung in unserer Anordnung sinnbehaftet ist.
Da wir in unserer Aufgabe alle Möglichkeiten der ersten 3 Plätze von 10
Teilnehmern darstellen wollen, ist eine Ordnung gegeben. Somit können
wir die \textbf{Kombination ausschließen}. Da jeder Teilnehmer nur
einmal teilnimmt, kann eine Wiederholung ebenfalls ausgeschlossen
werden.

Somit können wir \textbf{Definition 1.6} anwenden mit \[\begin{align}
n&=10 \\
k&=3
\end{align}\] folgende Lösung \[\begin{alignat}{3} 
V_{n}^k &= n(n-1)\dots(n-k+1) &&= n^{\underline{k}} &&= \frac{n!}{(n-k)!}\\
V_{10}^3 &= 10 \cdot 9 \cdot 8 &&= 10^{\underline{3}} &&= \frac{10!}{(10-3)!}\\
&&&&&=720
\end{alignat}\]
\end{quote}

\paragraph{Aufgabe (1d)}\label{aufgabe-1d}

\begin{quote}
Auch in dieser Aufgabe suchen wir nicht Anordnungen aus einer
Gesamtmenge. Wir können den Sachverhalt wie folgt darstellen. Wir haben
Aufgaben 1-10. Jede Aufgabe kann mit einer Antwort \(A,B,C\) angegeben
werden. Da nur eine der Aufgaben richtig ist, sind die Objekte wohl
unterscheidbar und können durch ihre Reihenfolge unterschieden werden.
\textbf{Für eine Aufgabe haben wir eine Variation ohne Wiederholung} mit
\[\begin{align}
n&=3 \\
k&=1
\end{align}\] und erhalten \[\begin{align}
V^k_{n}&=\frac{3!}{(3-1)!} \\
&= \frac{6}{2} \\
&= 3
\end{align}\] Das Ergebnis bestätigt auch die offensichtliche Eingebung,
dass wenn man eins aus drei ziehen kann, dass es nur 3 mögliche,
verschiedene Ausgänge geben kann.

Da in dem Wissenswettstreit diese Auswahl 10 mal vorgenommen wird,
erhalten wir \[3^{10}= 59\,049\] \textgreater{} Ergänzung für die
Anwendung von \(3^{10}\): \textgreater{} Nehmen wir an wir hätten 2
statt 10 Aufgaben. Wir fixieren für die erste Aufgabe Antwort \(A\). So
können wir unter der Auswahl von Antwort \(A\) in der ersten Aufgabe 3
verschiedene weitere Antworten für die zweite Aufgabe wählen. Also
\textgreater{} \[AA, AB, AC\] \textgreater{} Wenn wir für die erste
Aufgabe die Antwort \(B\) fixieren, können wir ein ähnliches Verhalten
beobachten, also \textgreater{} \[BA, BB, BC\] \textgreater{} Da für die
erste Aufgabe 3 Antworten möglich sind, und jeweils 3 weitere Antworten
folgen können, haben wir für zwei Aufgaben \textgreater{}
\[\begin{matrix}
AA & AB & AC \\
BA &BB & BC \\
CA & CB & CC
\end{matrix}\] , was einer quadratischen Darstellung gleicht, also
\(3^2\). Würden wir einen dritten Buchstaben hinzufügen, müssten wir für
jede Möglichkeit 3 weitere Möglichkeiten hinzufügen. Da es \(3^2\)
Möglichkeiten bereits sind, wären es demnach \(3^2 \cdot 3 = 3^3\).
Daher können wir in der Aufgabe ableiten, dass die Anzahl der Aufgaben,
also die Anzahl wie oft wir die Variation durchführen, den Exponenten
für die Anzahl der Variationen ergibt.
\end{quote}

\paragraph{Aufgabe 1(e)}\label{aufgabe-1e}

\begin{quote}
Wir haben insgesamt 8 Spieler zur Verfügung, wenn wir Spieler 1 mit den
verbleibenden 7 Spielern. Paaren wir nun Spieler 2, ist hier nun bereits
das Spiel mit Spieler 1 erfasst, also verbleiben für diesen nur noch 6
Spieler. Setzen wir das so fort, verbleiben für Spieler \(n\) immer noch
\(n-1\) weitere Spieler.

Daher sind es \[\begin{align}
\sum_{k=1}^{n-1}k & = \frac{n(n-1)}{2}  \\
 \sum_{k=1}^7 k &= 21
\end{align}\]

\begin{quote}
Wir haben Gebrauch der Gaußschen Summenformel gemacht, die da lautet
\[\sum_{k=1}^n k = \frac{n(n+1)}{2}  \] und statt \(n\) haben wir
\(n-1\).
\end{quote}
\end{quote}

\paragraph{Aufgabe 1(f)}\label{aufgabe-1f}

\begin{quote}
Zunächst bemerken wir, die 6 Objekte sind nicht unterscheidbar, und
somit gibt es keine Reihenfolge, die relevant ist. Daher können wir
bereits auf eine Kombination schließen. Da wir eine Wiederholung der
nicht unterscheidbaren Elemente erwarten, handelt es sich um eine
Kombination mit Wiederholung. Deshalb verwenden wir Definition 1.18 mit
\[\begin{align}
n &= 12 \\
k &= 6  
\end{align}\] Also haben wir \[\begin{align}
^wC^k_{n} &= \binom{n+k-1}{k} \\
^wC^6_{12} &= \binom{17}{6}  \\
&= \frac{17!}{(17-6)!} \\
&= 8\;910\;720
\end{align}\]
\end{quote}

\paragraph{Aufgabe 1(g)}\label{aufgabe-1g}

\begin{quote}
Da wir eine Auswahl treffen, statt eine Anordnung durchzuführen, spielt
die Reihenfolge keine Rolle und wir arbeiten mit der Kombination. Die
jeweiligen Kartenkategorien sind durch ihre Zugehörigkeit zu Herz, Pik,
Karo und Kreuz unterscheidbar. Wir werden zunächst jeweils die
Möglichkeiten die jeweiligen Elemente auszuwählen untersuchen.

Da jeweils die Damen/Asse/Luschen unterscheidbar sind, nur ihre
Reihenfolge keine Rolle spielt. Wir wählen wie folgt aus - 2 aus 4 Damen
=\textgreater{} \(n=4\) und \(k=2\) - 1 aus 4 Asse =\textgreater{}
\(n=4\) und \(k=1\) - 2 aus 36 Luschen =\textgreater{} \(n=36\) und
\(k=2\)

und erhalten: 1. Wahl der Damen \[\begin{align}
C^k_{n} &= \binom{n}{k} = \frac{n!}{n!(n-k)!} \\
C^2_{4} &= \binom{4}{2} = \frac{4!}{2!\cdot2!} \\
    &= 6
\end{align}\] 2. Wahl der Asse: \[\begin{align}
C^1_{4} &= \binom{4}{1} = \frac{4!}{(4-1)!} \\
&=4
\end{align}\] 3. Wahl der Luschen: \[\begin{align}
 C^2_{36} &= \binom{2}{36} = \frac{36!}{2!(36-2)!} \\
&= 630
\end{align}\]

Um die gesamte Möglichkeiten zu erhalten, müssen wir jede der
Möglichkeiten mit denen der anderen verknüpfen. Dies ist als das
Multiplikationsprinzip bekannt und wie der Name andeutet multiplizieren
wir die einzelnen Ergebnisse und erhalten
\[6 \cdot 4 \cdot 630 = 15\;120\]
\end{quote}

\paragraph{Aufgabe 1(h)}\label{aufgabe-1h}

\begin{quote}
Um die Aufgabe effektiv zu lösen, teilen wir die Menge der Ziffern in
die, die wir ohne Bedingung anordnen können und diese, die unter
Bedingung platziert werden - Mit Bedingung: \(M_{1} = \{ 1,3,5 \}\) -
Ohne Bedingung: \(M_{2} = \{ 1,2,4,6 \}\)

Den ersten Schritt können wir unabhängig von der Teilaufgabe
durchführen. Für jede Variante, die wir ermitteln, können die Elemente
aus \(M_2\) (ohne Bedingungen) verschiedenartig angeordnet werden, ohne,
dass es unsere Bedingung verletzt. Also erhalten wir eine Permutation
für \(M_2\) mit \(n=4\) \[\begin{align}
P_{4} &= 4! \\
   &= 24
\end{align}\] Nun lösen wir für \(M_1\) unter den verschiedenen
Teilaufgaben 1. \((1,3,5)\) werden nebeneinander in gegebener
Reihenfolge platziert Wir stellen uns eine Reihenfolge der Zahlen aus
\(M_{2}\) vor, beispielsweise: \[\begin{align}
|\; 4 \; | \; 2 \; | \;1\; | \;6\; |   
\end{align}\] Bemerkt die vertikalen Striche, die als Platzhalter für
das Paar \((1,3,5)\) stehen. Wir fragen nun, wieviele Möglichkeiten gibt
es dieses Paar dort einzufügen. Genau genommen handelt es sich hier um
eine Variation, in der wir einen der Positionen aus 5 wählen.
Offensichtlich, auch ohne Berechnungen zu bemühen, sehen wir, dass wir
lediglich 5 Möglichkeiten haben. \[V_{5}^1 = 5\,\]Nach dem
Multiplikationsprinzip multiplizieren wir das Ergebnis mit den
Permutationen aus unserer Vorbereitung und erhalten \[5 \cdot 24 = 120\]
2. \(\{ 1,3,5 \}\) werden nebeneinander aber in beliebiger Reihenfolge
platziert Wenn die Reihenfolge beliebig ist, suchen wir die
Permutationen unter den 3 gegebenen Elementen. Wir erhalten
\[P_{3} = 3! = 6\] Da diese Bedingung eine Erweiterung der Teilaufgabe
1. ist und für jede Möglichkeit in 1. jede Permutation aus 2. verwendet
werden kann, verwenden wir wieder das Multiplikationsprinzip und
erhalten \[5 \cdot 24 \cdot 6 = 720\]
\end{quote}

\subsubsection{Elementarereignisse}\label{elementarereignisse}

\paragraph{Aufgabe (2)}\label{aufgabe-2}

Um diese Aufgabe möglichst einfach lösen zu können, überlegen wir uns
wie die Elementarereignismenge \(\Omega\) aussieht. Die
Elementarereignismenge hängt vom vorgeschlagenen Zufallsexperiment ab.

Nehmen wir das Würfeln eines idealen Würfels, dann ist es unstrittig,
dass dieser 6 verschiedenen Augen hat. Wie die Elementarereignisse hier
aussehen hängt nun davon ab, wie wir definieren wie oft wir werfen.
Angenommen wir beschreiben das Zufallsexperiment ``zwei mal zu
würfeln'', dann würde der Gesamtereignisraum wie folgt aussehen

\[\begin{align}
 \Omega &= \{ (1,1), (1,2), \dots, (6,6) \} \\
&= \{ (i,j)\; | \; 1 \leq i,j \leq 6 \}
\end{align}\]

Haben wir uns auf eine Elementarereignismenge festgelegt, prüfen wir, ob
die vorgeschlagenen Ereignisse dort enthalten ist. Wir können auch immer
prüfen, dass die Elementarereignisse sich untereinander ausschließen
ausschließen und somit Definition 2.2 erfüllen. \#\#\#\# (2a)
\textgreater{} \textgreater{} Zunächst kodieren wir die möglichen
Beobachtungen im Zufallsexperiment \textgreater{} \[\begin{align}
> N &\; \widehat{ = } \text{ Nicht-Treffer}  \\
> T &\; \widehat{ = } \text{ Treffer}  \\
>\end{align}\] \textgreater Da wir unser Experiment mit ``zwei
abgefeuerten Schüssen'' beschreiben, sieht die Ereignismenge wie folgt
aus \textgreater{}\[\Omega =\{ NN, NT, TN, TT \}\] \textgreater{}
\textgreater Für die weitere Analyse behelfen wir uns der Definition
2.3, und erkennen, dass Ereignisse \(A_1, A_2, A_3\) Mengen sein müssen.
Wir drücken diese somit wie folgt aus \textgreater{}\[\begin{align}
> A_{1} &= \{ N \} \\
> A_{2} &= \{ T \} \\
> A_{3} &= \{  TT \}
>\end{align}\] \textgreater Damit eines dieser Ereignisse ein
Elementarereignis ist, muss es in \(\Omega\) enthalten sein, wir prüfen
nun \textgreater{}\[\begin{align}
> A_{1} \not\subseteq \Omega \quad &\Longrightarrow \quad \text{ kein Elementarereignis} \\
> A_{2} \not\subseteq \Omega \quad &\Longrightarrow \quad \text{ kein Elementarereignis} \\
> A_{3} \subseteq \Omega \quad &\Longrightarrow \quad \text{ Elementarereignis}
>\end{align}\]

\paragraph{(2b)}\label{b}

\begin{quote}
Wieder einigen wir uns auf eine sinnvolle Kodierung der Karten. Wir
haben die Karten - der Kategorie \(7-10\) und Bube, Dame König Ass, und
- der Kategorie der Farbe, die wir auf rot und schwarz reduziert haben

Wir kodieren die Begriffe wie folgt \[\begin{align}
7\text{-}10 &\; \widehat{ = } 7\text{-}10   \\
B &\; \widehat{ = } \text{ Bube}  \\
D &\; \widehat{ = } \text{ Dame}  \\
K &\; \widehat{ = } \text{ König}  \\
A &\; \widehat{ = } \text{ As}  \\
R &\; \widehat{ = } \text{ Rot}  \\
S &\; \widehat{ = } \text{ Schwarz}  \\
\end{align}\] Eine Karte beschreiben wir als Tupel bestehend aus (Wert,
Farbe). Beispiel, der schwarze Bube wäre demnach: \[(B, S)\] oder die
rote 7 wäre \[(7, R)\] Wir beschreiben die Menge der Werte durch \(W\)
und die Menge der Farben durch \(F\). Das Zufallsexperiment wird
beschrieben durch ``das Ziehen zweier Karten'', und kann daher als
folgende Ereignismenge dargestellt werden \[\begin{align}
\Omega &= \Big\{\{ (w_{1},f_{1}), (w_{2},f_{2}) \} \; \Big| \; w_{1} \neq w_{2} \lor \;f_{1} \neq f_{2}, \;\; w_{1},w_{2} \in W, f_{1},f_{2} \in F  \Big\} \\ \\
&= \Big\{ \{ (7,R), (7, S) \}, \dots, \{(A, R), (A,S)  \} \Big\}
\end{align}\] Um die folgenden Ereignisse zu beschreiben, einigen wir
uns darauf für einen beliebigen Wert einer Kategorie das Symbol \(x\) zu
verwenden. Also wenn wir schreiben \((x, R)\) ist jede der 8 roten
Karten möglich. \[\begin{align}
A_{1} &= \Big\{ (x_{1}, R), (x_{2}, R) \Big\} \\
A_{2} &= \Big\{ (x_{1}, S), (x_{2}, S)\Big\} \\
A_{3} &= \Big((x_{1}, R), (x_{2}, S)\Big) \\
A_{4} &= \Big\{ (x_{1}, R), (x_{2},  S) \Big\}
\end{align}\] Unter unserer Konstruktion von \(\Omega\) sind
\(A_1, A_{2}, A_{4}\) Elementarereignisse und \(A_3\) nicht, da
\[\begin{align}
A_{1} \subseteq \Omega \quad &\Longrightarrow \quad \text{Elementarereignis} \\
A_{2} \subseteq \Omega \quad &\Longrightarrow \quad \text{ Elementarereignis} \\
A_{3} \not\subseteq \Omega \quad &\Longrightarrow \quad \text{ kein Elementarereignis} \\
A_{4} \subseteq \Omega \quad &\Longrightarrow \quad \text{ Elementarereignis} \\
\end{align}\] \(A_{3}\) is dadurch kein Elementarereignis, das es durch
ein geordnetes Wertepaar beschrieben wird und die elemente in \(\Omega\)
ungeordnet, also Mengen sind.

Jedoch, da die Aufgabe nicht genau definiert, ob die vorgenommene
Auswahl geordnet oder ungeordnet ist, können wir die Elemente von
\(\Omega\) auch als geordnete Auswahl konstruieren, wodurch dann
\(A_1, A_{2}, A_{4}\) keine Elementarerignisse sind und wiederum \(A_3\)
ein Elementarereignis ist.
\end{quote}

\paragraph{Aufgabe 3}\label{aufgabe-3}

\begin{quote}
Für diese Aufgabe konstruieren wir zunächst die Ereignisse. Das
Zufallsexperiment beschreibt die Abgabe von \textbf{2 Schüssen}. Wir
kodieren wie zuvor die Beobachtungen als \[\begin{align}
N &\; \widehat{ = } \text{ Nicht-Treffer}  \\
T &\; \widehat{ = } \text{ Treffer}  \\
\end{align}\] und können die Ereignisse wie folgt beschreiben
\[\begin{align}
A_{1} &= \{ NN  \} \\
A_{2} &= \{  TN ,  NT  \} \\
A_{3} &= \{  NN ,  NT , TN \} \\
A_{4} &= \{ NT, TN, TT \}
\end{align}\] Über die Mengenoperation des Schnitts prüfen wir, ob die
Ereignisse paarweise disjunkt sind. \[\begin{alignat}{2}
A_{1} \cap A_{2} &=  \emptyset \quad &&\Longrightarrow \quad \text{disjunkt} \\
A_{1} \cap A_{3} &=  \{ NN \} \quad &&\Longrightarrow \quad \text{nicht disjunkt} \\ 
A_{1} \cap A_{4} &=  \emptyset \quad &&\Longrightarrow \quad \text{disjunkt} \\  
A_{2} \cap A_{3} &=  \{ NT, TN \} \quad &&\Longrightarrow \quad \text{nicht disjunkt} \\  
A_{2} \cap A_{4} &=  \{ NT, TN \} \quad &&\Longrightarrow \quad \text{nicht disjunkt} \\  
A_{3} \cap A_{4} &=  \{ NT, TN \} \quad &&\Longrightarrow \quad \text{nicht disjunkt} \\  
\end{alignat}\]
\end{quote}

\end{document}

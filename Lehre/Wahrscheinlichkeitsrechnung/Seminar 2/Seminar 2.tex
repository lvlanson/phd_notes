% Options for packages loaded elsewhere
\PassOptionsToPackage{unicode}{hyperref}
\PassOptionsToPackage{hyphens}{url}
%
\documentclass[
]{article}
\usepackage{amsmath,amssymb}
\usepackage{iftex}
\ifPDFTeX
  \usepackage[T1]{fontenc}
  \usepackage[utf8]{inputenc}
  \usepackage{textcomp} % provide euro and other symbols
\else % if luatex or xetex
  \usepackage{unicode-math} % this also loads fontspec
  \defaultfontfeatures{Scale=MatchLowercase}
  \defaultfontfeatures[\rmfamily]{Ligatures=TeX,Scale=1}
\fi
\usepackage{lmodern}
\ifPDFTeX\else
  % xetex/luatex font selection
\fi
% Use upquote if available, for straight quotes in verbatim environments
\IfFileExists{upquote.sty}{\usepackage{upquote}}{}
\IfFileExists{microtype.sty}{% use microtype if available
  \usepackage[]{microtype}
  \UseMicrotypeSet[protrusion]{basicmath} % disable protrusion for tt fonts
}{}
\makeatletter
\@ifundefined{KOMAClassName}{% if non-KOMA class
  \IfFileExists{parskip.sty}{%
    \usepackage{parskip}
  }{% else
    \setlength{\parindent}{0pt}
    \setlength{\parskip}{6pt plus 2pt minus 1pt}}
}{% if KOMA class
  \KOMAoptions{parskip=half}}
\makeatother
\usepackage{xcolor}
\setlength{\emergencystretch}{3em} % prevent overfull lines
\providecommand{\tightlist}{%
  \setlength{\itemsep}{0pt}\setlength{\parskip}{0pt}}
\setcounter{secnumdepth}{-\maxdimen} % remove section numbering
\ifLuaTeX
  \usepackage{selnolig}  % disable illegal ligatures
\fi
\IfFileExists{bookmark.sty}{\usepackage{bookmark}}{\usepackage{hyperref}}
\IfFileExists{xurl.sty}{\usepackage{xurl}}{} % add URL line breaks if available
\urlstyle{same}
\hypersetup{
  hidelinks,
  pdfcreator={LaTeX via pandoc}}

\author{}
\date{}

\begin{document}

\subsubsection{Klassische
Wahrscheinlichkeit}\label{klassische-wahrscheinlichkeit}

\paragraph{Aufgabe (1a)}\label{aufgabe-1a}

\begin{quote}
Um diese Aufgabe vernünftig zu lösen bedienen wir uns Definition 3.6 und
3.7. Um Definition 3.6 vernünftig anzuwenden, ist die Anzahl der
möglichen Elementarereignisse notwendig. Im letzten Seminar haben wir
dazu Übungen bereits durchgeführt.

Wir bemerken, es werden 3 ideale Würfel geworfen. Da die Reihenfolge
hierbei keine Rolle spielt, können wir die Elementarereignismenge wie
folgt angeben \[\begin{align}
\Omega &= \Big\{ \{a, b, c  \} \;\Big|\; 1 \leq a, b,c \leq 6 \Big\} \\
  &= \Big\{ \{1, 1, 1\}, \{ 1,1,2 \}, \{ 1,2,2 \}, \dots, \{ 6,6,6 \} \}
\end{align}\] , wobei die innere Menge eine \textbf{Multimenge} ist
(Definition 1.3). Da Mengen (auch Multimengen) keine Reihenfolge
besitzen und wir uns überlegen müssen wieviele Elemente in der
Elementarereignismenge \(\Omega\) vorhanden sind, verwenden wir unsere
kombinatorischen Fertigkeiten.

Die suchen daher wie wir 3 aus 6 mit Wiederholung wählen können, was die
\textbf{Kombination mit Wiederholung} ist (Definition 1.18). Wir haben
also \[\begin{align}
^wC_{n}^k &= \binom{n+k-1}{k} \\
^wC_{6}^3 &= \binom{6+3-1}{3} \\
    &= \frac{8!}{3!(8-3)!} \\
    &= 56
\end{align}\] Laut Definition 3.6 hat jedes einzelne Elementarereignis
\(\omega_{i}\) nun die Wahrscheinlichkeit
\[P(\omega_{i}) = \frac{1}{56}\] Wir lösen nun die Teilaufgaben mithilfe
von Definition 3.7 1. ``Wir würfeln genau eine 6'' beschreibt die
Ereignismenge \(A_{1}\) Hier haben wir in der Ereignismenge alle
Ereignisse \(A_1\), in der wir genau eine 6 finden. Da die Reihenfolge
keine Rolle spielt, suchen wir lediglich alle Kombinationen der anderen
zwei Würfel. Daher können wir die 6 als gegeben voraussetzen und suchen
für die anderen Würfel die Kombinationen ohne 6. Wir formulieren die
Kombination mit Wiederholung 2 aus 5 (ohne die 6), also \[\begin{align}
\lvert A_{1} \rvert &= \,^wC_{n}^k \\
    &= C_{6}^2 \\
    &= \binom{5+2-1}{2} \\
    &= \frac{6!}{2!(6-2)!} \\
    &= 15
\end{align}\] Nach gegebener Definition errechnet sich die
Wahrscheinlichkeit für \(P(A_{1})\) nun wie folgt \[\begin{align}
P(A_{1}) &= \frac{\lvert A_{1} \rvert }{\lvert \Omega \rvert }   \\
     &= \frac{15}{56} \\
     &\approx 0.268
\end{align}\] \textgreater{} Bemerkung: Wir verwenden hier die
Kardinalität \(\lvert \cdot \rvert\) der Menge \(A_{1}\), also wir
Zählen alle Elemente in der Menge. In Definition 3.7 haben wir
\[\begin{align}
> \lvert A \rvert &= n  \\
> \lvert \Omega \rvert &= m  
>\end{align}\] 1. ``Wir würfeln mindestens eine 6'' beschreibt die
Ereignismenge \(A_2\) Wir stehen nun vor derselben Aufgabe die Elemente
der Ereignismenge \(A_2\) zu zählen. Da wir in jedem Fall eine 6 haben,
interessieren uns wieder nur die anderen 2 Würfel und wir einigen uns
darauf, einen Würfel mit einer 6 zu fixieren. Jetzt können wir für die 2
verbleibenden würfel alle beliebigen Kombinationen angeben, also wir
wählen 2 aus 6 mit Wiederholung \[\begin{align}
\lvert A_{2} \rvert &= \;^wC^k_{n}  \\
       &= \,^wC^2_{6} \\
       &= \binom{6+2-1}{2} \\
    &= \frac{7!}{2!(7-2)!} \\
    &= 21
\end{align}\] Wie in der Aufgabe zuvor können wir jetzt die
Wahrscheinlichkeit berechnen \[\begin{align}
P(A_{2}) &= \frac{\lvert A_{2} \rvert}{\lvert \Omega \rvert } \\
     &= \frac{21}{56}\\
    &=0.375
\end{align}\]
\end{quote}

\paragraph{Aufgabe (1b)}\label{aufgabe-1b}

\begin{quote}
WIe in der Aufgabe zuvor sind wir mit dem Zählen der
Elementarereignismenge \(\Omega\) und der Ereignismenge \(A\)
konfrontiert. \#\#\#\#\#\# Elementarereignismenge Da die letzten 3
Ziffern in einer Ordnung stehen und wir aus 10 verschiedenen anordnen
können, handelt es sich hierbei eine Variation. Weiterhin wissen wir,
dass die Ziffern unterschiedlich sind, deshalb kann es hier keine
Wiederholungen geben. Wir erhalten daher \[\begin{align}
\lvert \Omega \rvert &= \frac{n!}{(n-k)!} & \\
      &= \frac{10!}{7!} = 720
\end{align}\] Da nur eine Telefonnummer die richtige ist, ist für die
Ereignismenge die Anzahl gegeben als \[\lvert A \rvert = 1 \] Wir
erhalten die Wahrscheinlichkeit \[\begin{align}
P(A) &= \frac{\lvert A \rvert}{\lvert \Omega \rvert }  \\
&= \frac{1}{720} \\
&\approx 0.0014
\end{align}\]
\end{quote}

\paragraph{Aufgabe (1c)}\label{aufgabe-1c}

\begin{quote}
Die Anzahl der Elementarereignisse lässt sich hier leicht bestimmen, da
wir genau eine Karte von 52 ziehen, also haben wir genau 52 verschiedene
Elementarereignisse \[\lvert \Omega \rvert = 52 \] Wir beschreiben ab
jetzt alle Ereignismengen mit \(A_i\) 1. ``Die gezogene Karte ist eine
Karokarte'' beschreibt \(A_1\) Wir haben genau
\[\lvert A_{1} \rvert = \frac{52}{4} = 13\] Karokarten, also können wir
die Wahrscheinlichkeit angeben \[\begin{align}
P(A_{1}) &= \frac{13}{52}  \\
&= 0.25
\end{align}\] 2. ``Die gezogene Karte ist Bube, Dame oder König''
beschreibt \(A_2\) Wir haben für jede der 4 Farben für jede dieser 3
Karten, daher können wir die Anzahl der Elementarereignisse angeben als
\[\lvert A_{2} \rvert  = 12\] und die Wahrscheinlichkeit \[\begin{align}
P(A_{2}) &= \frac{12}{52}  \\
&\approx 0.231
\end{align}\] 3. ``Die gezogene Karte ist eine Karokarte mit Bild (Bube,
Dame, König)'' beschreibt \(A_{3}\) Es gibt unter den gegebenen
Bedingungen genau 3 Karten, also \[\lvert A_{3} \rvert = 3 \] Also geben
wir die Wahrscheinlichkeit an als \[\begin{align}
P(A_{3}) &= \frac{3}{52} \\
     &\approx 0.058
\end{align}\]
\end{quote}

\subsubsection{Unabhängigkeit zufälliger
Ereignisse}\label{unabhuxe4ngigkeit-zufuxe4lliger-ereignisse}

\paragraph{Aufgabe (2)}\label{aufgabe-2}

\begin{quote}
Um festzustellen, ob Ereignisse unabhängig sind, verwenden wir
Definition 3.23 und 3.24. Um diese ordentlich zu verwenden ist es
notwendig, dass wir die entsprechenden Mengen konstruieren, die wir auf
Unabhängigkeit prüfen wollen. Unser Experiment wurd durch den Wurf einer
idealen Münze beschrieben. Daher kodieren wir \[\begin{align}
K &\;\widehat{=} \text{ Kopf} \\
Z &\; \widehat{=} \text{ Zahl}
\end{align}\] Wir bemerken zuvor, dass \[\begin{align}
\lvert \Omega \rvert&=P_{3}\\ &= 2^3 =8
\end{align}\] 1. \(A\)-``gleiche Seiten bei den beiden letzten Würfen.``
\[A = \{ (K,K,K), (Z,K,K), (K,Z,Z), (Z,Z,Z) \}\] 2. \(B\)-``gleiche
Seiten beim 1. und 3. Wurf``
\[B= \{ (K,Z,K), (K,K,K), (Z,K,Z), (Z,Z,Z) \}\] 3. \(C\)-``gleiche
Seiten bei den ersten beiden Würfen``
\[C= \{ (K,K,Z), (K,K,K), (Z,Z,K), (Z,Z,Z) \}\] Nun können wir die
Definition 3.19 anwenden \#\#\#\#\#\# 1. Teilaufgabe: Paarweise Lineare
Unabhängigkeit (Definition 3.23) \[\begin{align}
P(A \cap B) &= P\Big(\{ (K,K,K), (Z,Z,Z) \}\Big) \\
&= \frac{2}{8} \\
&= 0.25 \\
P(A)P(B) &= \frac{4}{8} \cdot \frac{4}{8} \\
&=0.25 \\
&\Rightarrow A \text{ und } B\text{ Linear Unabhängig} \\
P(A \cap C) &= P\Big(\{ (K,K,K), (Z,Z,Z) \}\Big) \\ 
&=\frac{2}{8} \\
&= 0.25 \\
P(A)P(C) &= 0.25 \\
&\Rightarrow A \text{ und } C\text{  Unabhängig} \\ 
P(A \cap C) &= P\Big(\{ (K,K,K), (Z,Z,Z) \}\Big) \\ 
&=\frac{2}{8} \\
&= 0.25 \\
P(A)P(C) &= 0.25 \\
&\Rightarrow B \text{ und } C\text{ Unabhängig} \\
\end{align}\] Demzufolge sind \(A,B,C\) paarweise unabhängig 2.
Teilaufgabe: Vollständige Unabhängigkeit Da wir für \(i=2\) in der
paarweise Unabhängigkeit alle Varianten schon geprüft haben, folgt nun
die Unabhängigkeit über alle Mengen \[\begin{align}
P\left( \bigcap_{k=1}^m A_{k} \right) &= P\Big(\{ (K,K,K), (Z,Z,Z) \}\Big) \\
&= 0.25 \\
\prod_{k=1}^m P(A_{k}) &= \left( \frac{1}{2} \right)^3 \\
           &= 0.125
\end{align}\] Demzufolge sind \(A,B,C\) \textbf{in ihrer Gesamtheit
nicht unbhängig}
\end{quote}

\subsubsection{Bedingte
Wahrscheinlichkeit}\label{bedingte-wahrscheinlichkeit}

\paragraph{Aufgabe (3)}\label{aufgabe-3}

\begin{quote}
In dieser Aufgabe werden wir die Erkenntnisse zu bedingter
Wahrscheinlichkeit benötigen. Wir strukturieren zunächst die
Informationen, die uns in der Aufgabe zur Verfügung stehen. Wir
formulieren folgende Szenarien - Eine Person wird zufällig gewählt,
handelt es sich dabei um einen Mann oder eine Frau? =\textgreater{}
Ereignis \(F\) kennzeichnet es ist \textbf{eine Frau} =\textgreater{}
Ereignis \(\lnot F\) kennzeichnet es ist \textbf{ein Mann} - Eine Person
wird zufällig gewählt, ist die Person farbenblind? =\textgreater{}
Ereignis \(B\) kennzeichnet die Person ist \textbf{farbenblind}
=\textgreater{} Ereignis \(\lnot B\) kennzeichnet die Person ist
\textbf{nicht farbenblind}

Wir haben folgende Wahrscheinlichkeiten gegeben: \[\begin{align}
P(F) &= 0.5 \\
P(M) &= 0.5\\
P(B\;|\;F) &= \frac{2}{1000}  \\
   &= 0.002 \\
P(B\;|\;M) &= \frac{5}{100} \\
&= 0.05
\end{align}\] Es ist gefragt unter der Bedingung, dass eine Person
farbenblind ist, wie hoch ist die Wahrscheinlichkeit, dass diese ein
Mann ist. Wir können dies mathematisch wie folgt darstellen
\[P(M\;|\;B)\] Was uns fehlt, um die Aufgabe zu lösen ist die
Wahrscheinlichkeit \(P(B)\). Diese erhalten wir, indem wir den
\textbf{Satz von der Totalen Wahrscheinlichkeit (3.28)} anwenden.
\[\begin{align}
P(B) &= \sum_{i=1}^n P(B\;|\;A_{i})P(A_{i}) \\
P(B) &=P(B\;|\;F)\cdot P(F) + P(B\;|\;M) \cdot P(M) \\
     &= 0.002 \cdot 0.5 + 0.05 \cdot 0.5 \\
     &= 0.026
\end{align}\] Jetzt können wir lösen, indem wir uns Satz 3.16 bedienen,
der sagt \[\begin{align}
P(A\;|\;B) &=\frac{P(A)}{P(B)}P(B\;|\;A)
\end{align}\] Wir setzen entsprechend ein und erhalten \[\begin{align}
P(M\;|\;B) &=\frac{P(M)}{P(B)}P(B\;|\;M) \\
  &= \frac{0.5}{0.026} \cdot 0.05 \\
&= 0.962
\end{align}\]
\end{quote}

\paragraph{Aufgabe (4)}\label{aufgabe-4}

\begin{quote}
Wir verfahren wie in der Aufgabe zuvor und strukturieren zunächst das
gegebene Wissen. Wir bezeichnen - \(T\) - ist ein Treffer - \(\lnot T\)
- Ist kein Treffer - \(G_{i}\) - wählt Gewehr \(i\) aus Wir halten
folgende Wahrscheinlichkeiten fest \[\begin{align}
P(T\;|\;G_{1})&= 0.5 \\
P(T\;|\;G_{2})&= 0.6 \\
P(T\;|\;G_{3})&= 0.7 \\
P(T\;|\;G_{4})&= 0.8 \\
P(T\;|\;G_{5})&= 0.9 \\
P(G_{i}) &= \frac{1}{5} =0.2 \qquad 1\leq i \leq 5
\end{align}\] 1. Teilaufgabe: Wir suchen \[P(T)\] Schauen wir die
gegebenen Größen an lässt sich die Aufgabe mit dem \textbf{Satz von der
Totalen Wahrscheinlichkeit (3.28)} lösen, also \[\begin{align}
P(T) &= \sum_{i=1}^5 P(T\;|\;G_{1}) \cdot P(G_{1}) \\
 &=0.5 \cdot 0.2 + 0.6 \cdot 0.2 + 0.7 \cdot 0.2 + 0.8 \cdot 0.2 +  0.9 \cdot 0.2 \\
&= 0.2 \cdot (0.5 + 0.6 + 0.7 + 0.8 + 0.9) \\
&= 0.7
\end{align}\] 2. Teilaufgabe: Wir suchen \[P(G_{5}\;|\;T)\] Dazu können
wir Satz 3.16 verwenden und erhalten \[\begin{align}
P(G_{5}\;|\;T)&= \frac{P(G_{5})}{P(T)} \cdot P(T\;|\;G_{5})\\
&= \frac{0.2}{0.7}\cdot 0.9 \\
&\approx 0.257
\end{align}\]
\end{quote}

\paragraph{Aufgabe (5)}\label{aufgabe-5}

\begin{quote}
Wir folgen derselben Prozedur wie in den 2 vorangegangenen Aufgaben und
kennzeichnen - \(E\) - Tuberkulose wird erkannt - \(\lnot E\) -
Tuberkulose wird nicht erkannt - \(K\) - ist an Tuberkulose erkrankt -
\(\lnot K\) - ist nicht an Tuberkulose erkrankt

Wir fassen die gegebenen Wahrscheinlichkeiten zusammen \[\begin{align}
P(E\;|\;K) &= 0.9 \\
P(E\;|\;\lnot K) &= 0.01  \\
P(K) &= 0.001 \\
P(\lnot K) &= 0.999
\end{align}\] Gesucht wird \[\begin{align}
P(K\;|\;E) 
\end{align}\] Um die gesuchte Größe mit \textbf{Satz 3.16} zu finden,
benötigen wir \(P(E)\). Dies können wir mit dem \textbf{Satz von der
Totalen Wahrscheinlichkeit (3.28)} finden, also \[\begin{align}
P(E) &= P(E\;|\;K) P(K) + P(E\;|\;\lnot K)P(\lnot K) \\
&= 0.9 \cdot 0.001 + 0.01 \cdot 0.999 \\
&= 0.01089
\end{align}\] Letztlich lösen wir mit \textbf{Satz 3.16}, also
\[\begin{align}
P(K\;|\;E) &= \frac{P(K)}{P(E)} \cdot P(E\;|\;K) \\
&= \frac{0.001}{0.01089} \cdot 0.9 \\
&= 0.0826
\end{align}\]
\end{quote}

\end{document}

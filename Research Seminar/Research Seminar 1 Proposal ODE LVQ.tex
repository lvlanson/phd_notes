% Options for packages loaded elsewhere
\PassOptionsToPackage{unicode}{hyperref}
\PassOptionsToPackage{hyphens}{url}
%
\documentclass[
]{article}
\usepackage{amsmath,amssymb}
\usepackage{iftex}
\ifPDFTeX
  \usepackage[T1]{fontenc}
  \usepackage[utf8]{inputenc}
  \usepackage{textcomp} % provide euro and other symbols
\else % if luatex or xetex
  \usepackage{unicode-math} % this also loads fontspec
  \defaultfontfeatures{Scale=MatchLowercase}
  \defaultfontfeatures[\rmfamily]{Ligatures=TeX,Scale=1}
\fi
\usepackage{lmodern}
\ifPDFTeX\else
  % xetex/luatex font selection
\fi
% Use upquote if available, for straight quotes in verbatim environments
\IfFileExists{upquote.sty}{\usepackage{upquote}}{}
\IfFileExists{microtype.sty}{% use microtype if available
  \usepackage[]{microtype}
  \UseMicrotypeSet[protrusion]{basicmath} % disable protrusion for tt fonts
}{}
\makeatletter
\@ifundefined{KOMAClassName}{% if non-KOMA class
  \IfFileExists{parskip.sty}{%
    \usepackage{parskip}
  }{% else
    \setlength{\parindent}{0pt}
    \setlength{\parskip}{6pt plus 2pt minus 1pt}}
}{% if KOMA class
  \KOMAoptions{parskip=half}}
\makeatother
\usepackage{xcolor}
\setlength{\emergencystretch}{3em} % prevent overfull lines
\providecommand{\tightlist}{%
  \setlength{\itemsep}{0pt}\setlength{\parskip}{0pt}}
\setcounter{secnumdepth}{-\maxdimen} % remove section numbering
\ifLuaTeX
  \usepackage{selnolig}  % disable illegal ligatures
\fi
\IfFileExists{bookmark.sty}{\usepackage{bookmark}}{\usepackage{hyperref}}
\IfFileExists{xurl.sty}{\usepackage{xurl}}{} % add URL line breaks if available
\urlstyle{same}
\hypersetup{
  hidelinks,
  pdfcreator={LaTeX via pandoc}}

\author{}
\date{}

\begin{document}

\subsection{1. Introduction to Linear Ordinary Differential
Equations}\label{introduction-to-linear-ordinary-differential-equations}

\begin{quote}
{[}!Question{]} Question: What is a Linear Ordinary Differential
Equation (ODE)? A linear ODE is an equation of the form
\[y^{(n)} =\sum_{i=0}^{n-1} a_{i}y^{(i)}\] where \(y^{(i)}\) is the
\(i\)-th derivative of the function real/complex valued function
\(y(t)\) and \(a_i\) are either constants or functions of \(t\). In this
definition \(n\) gives the order of the ODE.

In an ODE the variable of interest is the unknown function \(y\).
\end{quote}

\begin{quote}
{[}!example{]} Example Linear First Order ODE ODE: \(y'=2y+3\) Solution:
\(y = ce^{2t}-\frac32\) \textgreater{[}!Calculation{]}- Calculation
(Separation of Variables)
\textgreater{}\[\begin{align} \frac{dy}{dt} &= 2y + 3 \\ 
>\frac{dy}{2dt} &= y + \frac{3}{2} \\
>\frac{1}{y+\frac{3}{2}}\; dy&= 2\; dt \\
>\int \frac{1}{y+\frac32} \; dy &= \int 2 \; dt \\
>\ln\left(y+\frac32\right) &= 2t +c_0\\
>y + \frac32&=e^{2t} e^{c_0}\\
>y &= ce^{2t}-\frac32
>\end{align}\]

Note, solving an ODE yields an integration constant \(c\). Any choice of
\(c\) will be a solution to the ODE.

!{[}{[}ODE\_example\_1.png \textbar{} center \textbar{} 500 {]}{]}
\[\text{Figure: Direction Field of ODE with solutions at } y_{0}=(-20, -15, -10, -5, 0, 5, 10, 15) \text{ with } y_{0} = y(0)\]
\end{quote}

\begin{quote}
{[}!remark{]} The solution for a first order ODE characterizes
infinitely many functions, dependent on the choice of the integration
constant \(c\).
\end{quote}

\begin{quote}
{[}!theorem{]} Hypothesis: ODE LVQ Fundamental Principle 1. Let an ODE
represent a prototype in an LVQ learning scheme 2. Let an (for now)
arbitrary function represent a data point in the LVQ learning scheme

\textbf{Main Idea:} As previously remarked, an ODE \textbf{represents a
set of functions prototyped by the ODE}. What if we can develop a
learning scheme, such that * an ODE will be learned discriminating
functions
\end{quote}

\begin{quote}
{[}!Warning{]} Issues 1. How to initialize prototypes? 2. Given an ODE,
how to determine which function out of the set of functions described by
the ODE fits best to a datapoint (function)? 3. Time series data more
often than not is a set of discrete points. How should we get an ODE
from such a set of discrete data points? 4. How to compare an ODE
(prototype) to another function (data point)? (Metric) 5. Given we
solved the first issues, how to formulate an update rule?
\end{quote}

\subsection{2. Road to ODE LVQ}\label{road-to-ode-lvq}

\subsubsection{Issue 1 Prototype
Inititialization}\label{issue-1-prototype-inititialization}

\begin{quote}
{[}!theorem{]} Initialization of Prototypes 1. Choose an arbitrary
datapoint for a given class. 2. An ODE can be calculated given there is
a function suitable for calculating its ODE
\end{quote}

\begin{quote}
{[}!example{]} Example: Determining the ODE for a function
\(y(t)=\frac{c_{1}}{t}+c_{2}t\) First, we take the derivative with
respect to \(t\) \[\begin{align}
y' = -\frac{c_{1}}{t^2}+c_{2}
\end{align}\] Solving for \(c_2\) gives \[c_{2} = \frac{c_{1}}{t^2}+y'\]
Inserting this solution into the original equation gives \[\begin{align}
y &= \frac{c_{1}}{t}+ \left( \frac{c_{1}}{t^2}+y' \right)t \\
  &= \frac{c_{1}}{t}+ \frac{c_{1}}{t}+y't \\
  &= \frac{2c_{1}}{t}+y't \\
y-y't  &= 2c_{1}\\ 
yt-y't^2  &= 2c_{1} \\ 
\end{align}\] Taking now again the derivative with respect to \(t\)
yields \[\begin{align}
y+y't-2y't-y''t^2 = 0 \\
y-y't-y''t^2 = 0 \\
\end{align}\]
\end{quote}

\subsubsection{Issue 2 Function Choice in
ODE}\label{issue-2-function-choice-in-ode}

\begin{quote}
{[}!def{]} Definition Initial Value Problem
({[}{[}../../../Sources/nagy.pdf\#page=18\textbar Source{]}{]}) The
initial value problem (IVP) is to find all solutions \(y\) to
\[ y' = ay+b \] that satisfy the initial condition
\[ y(t_0) = y_0 \tag{Initial Condition}\] where \(a,b,t_0, y_0\) are
given constants. \textgreater{[}!note{]} \textgreater The differential
equation has a unique solution with respect to the initial condition.
\end{quote}

Similarly for second order ODEs etc. \textgreater{[}!theorem{]} Theorem
Solutions to IVP of Second Order ODEs
({[}{[}../../../Sources/nagy.pdf\#page=88 \textbar{} Source{]}{]})
\textgreater If \textgreater- the functions \(a_{0},a_{1}, b\) are
continuous on a closed interval \(I \subset \mathbb{R}\), \textgreater-
the constant \(t_{0} \in I\), \textgreater- and
\(y_{0}, y_{1} \in \mathbb{R}\) are arbitrary constants \textgreater{}
\textgreater then there is a unique solution \(y\), defined on \(I\), of
the initial value problem
\textgreater{}\[y'' + a_{1}(t)y'+ a_{0}(t)y = b(t)\] \textgreater with
\[ y(t_{0})=y_{0} \quad y'(t_{0})= y_{1}\]

\begin{quote}
{[}!remark{]} Remarks 1. Even if we would construct ODEs of higher
orders, we could still find a matching function in the higher order ODE
by utilizing the initial value problem. 2. Alex suggested that there
might be other ways to pick a matching ODE, so this can still be changed
if needed (needs more research) 3. This method for now allows a straight
forward approach to pick a function from the ODE
\end{quote}

\subsubsection{Issue 3 Discrete Data}\label{issue-3-discrete-data}

!{[}{[}Figures/rs\_1\_fig\_1.png\textbar center\textbar1000{]}{]}
\[\text{Figure: How to find an ODE representing the sequence of discrete points}\]

\begin{quote}
{[}!remark{]} Discussion: Many possibilities 1. Interpolation (Spline)
2. Possibly using Laplace transform and treat it as piecewise continuous
functions (not confirmed yet) 3. Dirac Delta Expansion with Fourier
Transform 4. My Approach: Discrete Fourier Transform (DFT)
\end{quote}

\begin{quote}
{[}!properties{]} Properties of Sine and Cosine Functions The
derivatives/anti-derivatives of \(\sin\) and \(cos\) behave nicely, i.e.
\[\begin{align}
\frac{d}{dt} \sin(ct) &= c \cos(ct) \\
\frac{d}{dt} \cos(ct) &= -c \sin(ct)
\end{align}\] \textgreater{[}!note{]} \textgreater The notion \emph{nice
behavior} means \(\sin\) and \(\cos\) are very predictable in their
derivatives/antiderivatives. The arguments inside the function stay
constant and, the derivatives switch between \(\sin\) and \(\cos\).
\end{quote}

\begin{quote}
{[}!example{]} Example Step Function Let's approximate the step function
the step function \[f(x) = \begin{cases}
1 & \text{ if } 0 \leq x < \pi \\
-1 & \text{ if } \pi \leq x < 2\pi
\end{cases}\] by a partial Fourier sum \(FS_{N}[f]_{t}\) with accuracy
\(N=2\), we get
\[y = \frac{4}{\pi}\left( \sin(x)+ \frac{1}{3}\sin(3x) + \frac{1}{5}\sin(5x)+ \frac{1}{7}\sin(7x) + \frac{1}{9}\sin(9x) \right)\]
We will omit the factor \(\frac{4}{\pi}\) for simplification purposes
\[y =  \sin(x)+ \frac{1}{3}\sin(3x) \]
!{[}{[}Figures/rs\_1\_fig\_2.png{]}{]}
\[\text{Figure: partial sum in green, step function in orange}\] Lets
determine an ODE from the approximation and parameterize the first
coefficient. Hence,
\[y = \underbrace{ c_{1} \sin(x) }_{ \substack{\text{ first}\\ \text{ coefficient}} }+ \frac{1}{3}\sin(3x) + \frac{1}{5}\sin(5x) + \frac{1}{7}\sin(7x) + \frac{1}{9}\sin(9x) \tag{1}\]
We determine the first derivative
\[y' = c_{1} \cos(x)+ \cos(3x)+ \cos(5x)+\cos(7x)+ \cos(9x)\] We solve
for \(c_{1}\)
\[c_{1} = \frac{y' - \cos(3x)-\cos(5x)-\cos(7x)-\cos(9x)}{\cos (x)}\]
Inserting \(c_{1}\) into equation \((1)\) gives \[\begin{align}
y &= \frac{(y' - \cos(3x)-\cos(5x)- \cos(7x)-\cos(9x))\sin (x)}{\cos (x)}+ \frac{1}{3}\sin(3x) + \frac{1}{5}\sin(5x)+ \frac{1}{7}\sin(7x) + \frac{1}{9}\sin(9x) \\
&=\frac{y'\sin(x)}{\cos(x)} -\frac{\sin(x)(\cos(3x)+ \cos(5x)+\cos(7x)+\cos(9x))}{\cos(x)} + \frac{1}{3}\sin(3x) + \frac{1}{5}\sin(5x)+ \frac{1}{7}\sin(7x) + \frac{1}{9}\sin(9x)
\end{align} \] We now need to solve for \(y'\) to yield the ODE, the
solution is \[\begin{align}
y' = \cos(3x)+\cos(5x)+\cos(7x)+\cos(9x) - \frac{\cos(x)}{\sin(x)}\left(\frac{1}{3}\sin(3x) + \frac{1}{5}\sin(5x)+\frac{1}{7}\sin(7x) + \frac{1}{9}\sin(9x) - y\right)
\end{align}\] !{[}{[}Figures/rs\_1\_fig\_3.png{]}{]}
\[\text{ODE with dependence on }\sin(x)\] If we solve for the \(4\)-th
coefficient (\(\frac{1}{7}\sin(7x)\)), we get
\[y' = \cos(x)+\cos(3x)+\cos(5x)+\cos(9x) - \frac{7\cos(7x)}{\sin(x)}\left(\sin(x)+\frac{1}{3}\sin(3x) + \frac{1}{5}\sin(5x)+ \frac{1}{9}\sin(9x) - y\right)\]
!{[}{[}Figures/rs\_1\_fig\_4.png{]}{]}
\[\text{ODE with dependence on } \frac{1}{7}\sin(7x)\]
\end{quote}

\begin{quote}
{[}!Observation{]} Dependent on which frequency term we use, the ODE
\end{quote}

\begin{quote}
{[}!question{]} Questions 1. Which parameter should be chosen to define
the ODE, and could it be beneficial to use higher order ODEs? 2. How
many should be chosen? 3. How would many parameters affect the
construction?
\end{quote}

\subsubsection{Issue 4 + 5 Comparing ODE to Function (Metric) and
updating
prototype}\label{issue-4-5-comparing-ode-to-function-metric-and-updating-prototype}

\begin{quote}
{[}!observation{]} Idea 1. Find function with \emph{initial value
problem} (IVP) 2. Compare IVP determined function with data point
function * using Dynamic Timewarping * using distance on \(L^2[a,b]\)
norm * using distance based on maximums norm 3. ??? * update IVP
function \(u^*\)-\textgreater{} determine ODE -\textgreater{} update
prototype with ODE from \(u^*\)
\end{quote}

\end{document}
